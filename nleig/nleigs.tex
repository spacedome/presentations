\documentclass[mathserif]{beamer}

\usepackage[utf8]{inputenc}
\usepackage{graphicx}
\usepackage{mathtools}
\usepackage{amsthm}
\usepackage{algorithmic}

\usepackage{pgfplots}
\pgfplotsset{compat=1.16}
\usepgfplotslibrary{colorbrewer}


\pgfplotsset{every axis/.append style={
            		width=2.5in,
            		grid=both,
                    label style={font=\footnotesize},
                    tick label style={font=\footnotesize},
            		cycle list/Set1-5,
            		cycle multiindex* list={
            		    Set1-5
            		    \nextlist
                        marks
                        \nextlist
                        [2 of]linestyles
                        \nextlist
            		    thick
            		    \nextlist
                    },
            }
}

\pgfplotsset{
    % define a `cycle list' for marker
    cycle list/.define={marks}{
	    every mark/.append style={solid,fill opacity=0.25,fill=\pgfkeysvalueof{/pgfplots/mark list fill}},mark=*\\
	    every mark/.append style={solid,fill opacity=0.25,fill=\pgfkeysvalueof{/pgfplots/mark list fill}},mark=square*\\
	    every mark/.append style={solid,fill opacity=0.25,fill=\pgfkeysvalueof{/pgfplots/mark list fill}},mark=triangle*\\
	    every mark/.append style={solid,fill opacity=0.25,fill=\pgfkeysvalueof{/pgfplots/mark list fill}},mark=diamond*\\
    },
}

\usetheme{default}
\usecolortheme{beaver}

\title{An Iterative Method for Contour Based Nonlinear Eigensolvers}
\author{Julien Brenneck \and Eric Polizzi}
\date{}
%\subject{Numerical Linear Algebra}

\beamertemplatenavigationsymbolsempty{}

%%%%%%%%%%%%%%%%%%%%%%%%%%%%%%%%%%%%%%%%%%%%%%%%%%%%%%%%%%%%%%%%%%%%%%%%%%%%%%%%
%%%%%%%%%%%%%%%%%%%%%%%%%%%%%%%%%%%%%%%%%%%%%%%%%%%%%%%%%%%%%%%%%%%%%%%%%%%%%%%%
%%%%%%%%%%%%%%%%%%%%%%%%%%%%%%%%%%%%%%%%%%%%%%%%%%%%%%%%%%%%%%%%%%%%%%%%%%%%%%%%

\begin{document}

\begin{frame}
 \titlepage{}
\end{frame}

%%%%%%%%%%%%%%%%%%%%%%%%%%%%%%%%%%%%%%%%%%%%%%%%%%%%%%%%%%%%%%%%%%%%%%%%%%%%%%%%
\section{Previous Techniques}
%%%%%%%%%%%%%%%%%%%%%%%%%%%%%%%%%%%%%%%%%%%%%%%%%%%%%%%%%%%%%%%%%%%%%%%%%%%%%%%%

\begin{frame}{Beyn's Method}

	Use Keldysh theorem to probe Jordan decomposition for spectral information, giving a linearization of \(T(\lambda)\).

	\begin{align}
		A_k = \frac{1}{2 \pi i} \int_\Gamma z^k {T(z)}^{-1} X \, dz 
		= V \Lambda^k W^H X
	\end{align}
	Taking the SVD \( A_0 = V_0 \Sigma_0 W_0^H \) we can derive the following (computable) similarity.
	\begin{align}
		\Lambda \sim V_0^H A_1 W_0 \Sigma_0^{-1}
	\end{align}
	This gives a standard problem \(  V_0^H A_1 W_0 \Sigma_0^{-1} Y = Y \Lambda\), taking \( X = V_0 Y  \) recovers eigenvectors.

\end{frame}

\begin{frame}{NLFEAST}

	Applying a Residual Inverse Iteration with contour points as fixed shifts gives a Newton-type iteration. 

	\begin{align}
		Q_0 = \frac{1}{2 \pi i } \int_\Gamma \Big(X - {T(z)}^{-1} T(X, \Lambda) \Big) {(zI - \Lambda)}^{-1} \, dz
	\end{align}
	Each iteration solve the projected problem.
	\begin{align}
		Q_0^H T(\lambda) Q_0 y = 0
	\end{align}
	Recover eigenvectors by taking \( X = Q_0 Y \).

\end{frame}

\begin{frame}{Problems}

	\begin{itemize}
		\item How to share contour nodes (and thus linear system solves) when using NLFEAST with Beyn?
		\vspace{2em}

		\item How can Beyn's method be iteratively refined? Doing so would address drawbacks of using many quadrature points.
	\end{itemize}

\end{frame}

%%%%%%%%%%%%%%%%%%%%%%%%%%%%%%%%%%%%%%%%%%%%%%%%%%%%%%%%%%%%%%%%%%%%%%%%%%%%%%%%
\section{Hybrid Algorithm}
%%%%%%%%%%%%%%%%%%%%%%%%%%%%%%%%%%%%%%%%%%%%%%%%%%%%%%%%%%%%%%%%%%%%%%%%%%%%%%%%

\begin{frame}{NLFEAST-Beyn Hybrid Algorithm}
	We can generalize the RII moment \( Q_0 \) of NLFEAST to \( Q_k \).	
	\begin{align}
		Q_k = \frac{1}{2 \pi i } \int_\Gamma z^k \Big(X - {T(z)}^{-1} T(X, \Lambda) \Big) {(zI - \Lambda)}^{-1} \, dz
	\end{align}
	Then apply the linearization of Beyn's method to these moments.

\end{frame}

\begin{frame}{NLFEAST-Beyn Hybrid Algorithm}

      \begin{algorithmic}
	%\REQUIRE{Initial (random) subspace \( X \in \mathbb{C}^{n \times m}\)}
	%\REQUIRE{Contour \( \Gamma \) and \( N \) quadrature nodes and weights \( (z_j, \omega_j) \)}
	%\REQUIRE{Stopping tolerence \( \epsilon \)}
	\STATE{\(Q_0 = \sum_{j=1}^{N} \omega_j {T(z_j)}^{-1} X \)}
	\STATE{\(Q_1 = \sum_{j=1}^{N} \omega_j z_j {T(z_j)}^{-1} X \)}
	\STATE{Compute the QR Decomposition \( qr \leftarrow Q_0\)}
	\STATE{\( B = q^H Q_1 r^{-1} \)}
	\STATE{Solve \( BY = Y \Lambda \)}
	\STATE{ \( X \leftarrow qY \)}
	\WHILE{not converged}
\STATE{\(Q_0 \leftarrow \sum_{j=1}^{N} \omega_j [X - {T(z_j)}^{-1}T(X, \Lambda)] {(z_j I - \Lambda)}^{-1} \)}
\STATE{\(Q_1 \leftarrow \sum_{j=1}^{N} \omega_j z_j [X  - {T(z_j)}^{-1} T(X, \Lambda)] {(z_j I - \Lambda)}^{-1} \)}
	\STATE{Compute the QR Decomposition \( qr \leftarrow Q_0\)}
	\STATE{\( B \leftarrow q^H Q_1 r^{-1} \)}
	\STATE{Solve \( BY = Y \Lambda \)}
	\STATE{ \( X \leftarrow qY \)}
	\ENDWHILE
	\RETURN \( X, \Lambda \)
      \end{algorithmic}
      
\end{frame}

\begin{frame}{NLFEAST-Beyn Hybrid Algorithm}
	\begin{itemize}
		\item From one perspective, this can be viewed as NLFEAST using the linearization (from Keldysh) of Beyn's method directly.
			\vspace{2em}
		\item From the other, it can be viewed as Beyn's method using the RII (from Neumaier) approach of NLFEAST.\@
			\vspace{2em}
		\item Can be reduced to standard FEAST for linear problems.
	\end{itemize}
\end{frame}

\begin{frame}{Higher Moments}
	
	Necessary to use higher moments for eigenvector defects or many eigenvalues in a contour.
	\vspace{1em}

	General way of considering most higher moment algorithms by considering ``left probing matrix'' \( \widehat{X} \in \mathbb{C}^{n\times \ell} \) in addition to ``right probing matrix'' \( X \in \mathbb{C}^{n\times m} \) and defining
	\begin{align}
		\widehat{A}_k = \widehat{X}^H A_k
		= \frac{1}{2 \pi i} \int_{\Gamma} z^k \widehat{X}^H {T(z)}^{-1} X \, dz.
	\end{align}


\end{frame}

\begin{frame}{Higher Moment Matrix}
	Choosing \( K \in \mathbb{N} \) as the number of computed moments, form the \( K \ell \times K m \) block Hankel matrices
	\begin{align}
		H_0 = 
		\begin{bmatrix}
			\widehat{A}_0      & \cdots & \widehat{A}_{K-1}  \\
			\vdots             &        & \vdots \\
			\widehat{A}_{K-1}  & \cdots & \widehat{A}_{2K-2} 
		\end{bmatrix},
		H_1 = 
		\begin{bmatrix}
			\widehat{A}_1      & \cdots & \widehat{A}_{K}  \\
			\vdots             &        & \vdots \\
			\widehat{A}_{K}    & \cdots & \widehat{A}_{2K-1} 
		\end{bmatrix}.
	\end{align}
	From the Keldysh theorem we have \( \widehat{A}_k = \widehat{X}V\Lambda^k W^H X \), letting
	\begin{align}
		V_{[K]} = 
		\begin{bmatrix}
			\widehat{X}V \\
			\vdots \\
			\widehat{X} V \Lambda^{K-1}
		\end{bmatrix},
		\quad
		W^H_{[K]} = \big[W^HX, \ldots, \Lambda^{K-1} W^H X \big]
	\end{align}
	we have, similarly to before with \( A_0\) and \(A_1 \), that 
	\begin{align}
		H_0 = V_{[K]}W^H_{[K]}, \quad
		H_1 = V_{[K]} \Lambda W^H_{[K]}.
	\end{align}
\end{frame}

\begin{frame}{Higher Moment Methods}
	Using the same similarity as Beyn's method, taking the SVD \( H_0 = V_0 \Sigma_0 W^H_0 \), we can diagonalize \( V_0^H H_1 W_0 \Sigma_0^{-1} \) to find \( \Lambda \).
	Recovering the eigenvectors depends on the form of \( \widehat{X} \).
	\vspace{1em}
	\begin{itemize}
		\item Taking \( \widehat{X} = I \in \mathbb{C}^{n \times n} \) gives Beyn's method.
			\vspace{1em}
		\item Taking \( \widehat{X} = X \) gives the SS-Hankel method.
			Here \( H_0, H_1 \) are square and another approach is need to recover eigenvectors.

	\end{itemize}

	\vspace{1em}
	For \( \widehat{X}=I \) (thus \( \widehat{A}_k = A_k \)) we have applied the RII method.
	This requires deflating the subspace after every iteration.

\end{frame}

\begin{frame}{Problems with Higher Moments}

	\begin{itemize}
		\item Eigenvectors (potentially) only recoverable in special cases, limiting iteration to Beyn and SS type methods.
			\vspace{1em}
		\item SS-Hankel approach has benefits, but not yet explored.
			\vspace{1em}
		\item Deflation after every iteration limits number of eigenvalues. Unclear how RII can be modified to reuse all spectral information after iteration.
			\vspace{1em}
		\item No way to determine when there are deficient eigenvectors without computing many higher moments.
			\vspace{1em}
		
	\end{itemize}

	
\end{frame}



%%%%%%%%%%%%%%%%%%%%%%%%%%%%%%%%%%%%%%%%%%%%%%%%%%%%%%%%%%%%%%%%%%%%%%%%%%%%%%%%
\section{Experimental Results}
%%%%%%%%%%%%%%%%%%%%%%%%%%%%%%%%%%%%%%%%%%%%%%%%%%%%%%%%%%%%%%%%%%%%%%%%%%%%%%%%

\begin{frame}{Numerical Experiments}
    \centering Hadeler Problem	
    \vspace{1em}
    \begin{figure}[htbp]
            \centering
	    \begin{tikzpicture}[scale=0.8]
            \begin{axis}[
                    legend style={font=\scriptsize, fill opacity=0.9},
                    unbounded coords=jump,
                    ymode=log,
                    xmax=10,
                    xmin=0.5,
                    xtick=data,
                    xlabel={Iteration},
                    ymin=1.0e-18,
                    ylabel={Residual},
                    extra x ticks={1},
                    extra x tick labels={{\scriptsize(Beyn)}},
                    extra x tick style={{yshift=-2ex, grid=none}},
                    legend entries={\(N=4\), \(N=8\), \(N=16\), \(N=32\), \(N=64\), \(N=128\), \(N=256\) } ]
                    \addplot table [x=iter, y=$2$] {hadeler.dat};
                    \addplot table [x=iter, y=$3$] {hadeler.dat};
                    \addplot table [x=iter, y=$4$] {hadeler.dat};
                    \addplot table [x=iter, y=$5$] {hadeler.dat};
                    \addplot table [x=iter, y=$6$] {hadeler.dat};
                    \addplot table [x=iter, y=$7$] {hadeler.dat};
                    \fill [blue, fill opacity=0.1] (0.5, 10.0) rectangle (1.5, 5.0e-19);
            \end{axis}
            \end{tikzpicture}
	    \begin{tikzpicture}[scale=0.8]
            \begin{axis}[
                    xlabel={Real},
                    ylabel={Imaginary},
                    xmin=-48,
                    xmax=-12,
                    ymin=-15,
                    ymax=15
                    ]
                    \draw[dashed, thick] (axis cs:-30.0,0.0) circle [radius=10.0];
                    \addplot+ [only marks] table {hadeler_eig_in.dat};
                    \addplot+ [only marks, mark=*, mark options={solid, scale=0.35}] table {hadeler_eig.dat};
            \end{axis}
            \end{tikzpicture}
    \end{figure}

\end{frame}

\begin{frame}{Numerical Experiments}

    \centering Butterfly Problem	
    \vspace{1em}

    \begin{figure}[htbp]
            \centering
	    \begin{tikzpicture}[scale=0.8]
            \begin{axis}[
                    ymode=log,
                    xmin=0.5,
                    xmax=13,
                    xtick=data,
                    unbounded coords=jump,
                    legend style={font=\scriptsize, fill opacity=0.9},
                    xlabel={Iteration},
                    ymin=5.0e-17,
                    ylabel={Residual},
                    ymin=5.0e-17,
                    extra x ticks={1},
                    extra x tick labels={{\scriptsize(Beyn)}},
                    extra x tick style={{yshift=-2ex, grid=none}},
                    legend entries={\(N=4\), \(N=8\), \(N=16\), \(N=32\), \(N=64\), \(N=128\), \(N=256\) } ]
                    \addplot table [x=iter, y=$2$] {bf.dat};
                    \addplot table [x=iter, y=$3$] {bf.dat};
                    \addplot table [x=iter, y=$4$] {bf.dat};
                    \addplot table [x=iter, y=$5$] {bf.dat};
                    \addplot table [x=iter, y=$6$] {bf.dat};
                    \addplot table [x=iter, y=$7$] {bf.dat};
                    \addplot table [x=iter, y=$8$] {bf.dat};
                    \fill [blue, fill opacity=0.1] (0.5, 10.0) rectangle (1.5, 5.0e-17);
            \end{axis}
            \end{tikzpicture}
            \begin{tikzpicture}[scale=0.8]
            \begin{axis}[
                    xlabel={Real},
                    ylabel={Imaginary},
                    xmin=0,
                    xmax=2,
                    ymin=0,
                    ymax=2]
                    \draw [dashed, thick] (axis cs:1.0,1.0) circle [radius=0.5];
                    \addplot+ [only marks] table {bf_eig_in.dat};
                    \addplot+ [only marks, mark=*, mark options={solid, scale=0.35}] table {bf_eig.dat};
            \end{axis}
            \end{tikzpicture}
    \end{figure}

  
\end{frame}

\end{document}
